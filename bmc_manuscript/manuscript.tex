%% BioMed_Central_Tex_Template_v1.06
%%                                      %
%  bmc_article.tex            ver: 1.06 %
%                                       %

%%IMPORTANT: do not delete the first line of this template
%%It must be present to enable the BMC Submission system to
%%recognise this template!!

%%%%%%%%%%%%%%%%%%%%%%%%%%%%%%%%%%%%%%%%%
%%                                     %%
%%  LaTeX template for BioMed Central  %%
%%     journal article submissions     %%
%%                                     %%
%%          <8 June 2012>              %%
%%                                     %%
%%                                     %%
%%%%%%%%%%%%%%%%%%%%%%%%%%%%%%%%%%%%%%%%%


%%%%%%%%%%%%%%%%%%%%%%%%%%%%%%%%%%%%%%%%%%%%%%%%%%%%%%%%%%%%%%%%%%%%%
%%                                                                 %%
%% For instructions on how to fill out this Tex template           %%
%% document please refer to Readme.html and the instructions for   %%
%% authors page on the biomed central website                      %%
%% http://www.biomedcentral.com/info/authors/                      %%
%%                                                                 %%
%% Please do not use \input{...} to include other tex files.       %%
%% Submit your LaTeX manuscript as one .tex document.              %%
%%                                                                 %%
%% All additional figures and files should be attached             %%
%% separately and not embedded in the \TeX\ document itself.       %%
%%                                                                 %%
%% BioMed Central currently use the MikTex distribution of         %%
%% TeX for Windows) of TeX and LaTeX.  This is available from      %%
%% http://www.miktex.org                                           %%
%%                                                                 %%
%%%%%%%%%%%%%%%%%%%%%%%%%%%%%%%%%%%%%%%%%%%%%%%%%%%%%%%%%%%%%%%%%%%%%

%%% additional documentclass options:
%  [doublespacing]
%  [linenumbers]   - put the line numbers on margins

%%% loading packages, author definitions

%\documentclass[twocolumn]{bmcart}% uncomment this for twocolumn layout and comment line below
\documentclass{bmcart}

%%% Load packages
%\usepackage{amsthm,amsmath}
\RequirePackage{natbib}
%\RequirePackage[authoryear]{natbib}% uncomment this for author-year bibliography
%\RequirePackage{hyperref}
\usepackage[utf8]{inputenc} %unicode support
%\usepackage[applemac]{inputenc} %applemac support if unicode package fails
%\usepackage[latin1]{inputenc} %UNIX support if unicode package fails


%%%%%%%%%%%%%%%%%%%%%%%%%%%%%%%%%%%%%%%%%%%%%%%%%
%%                                             %%
%%  If you wish to display your graphics for   %%
%%  your own use using includegraphic or       %%
%%  includegraphics, then comment out the      %%
%%  following two lines of code.               %%
%%  NB: These line *must* be included when     %%
%%  submitting to BMC.                         %%
%%  All figure files must be submitted as      %%
%%  separate graphics through the BMC          %%
%%  submission process, not included in the    %%
%%  submitted article.                         %%
%%                                             %%
%%%%%%%%%%%%%%%%%%%%%%%%%%%%%%%%%%%%%%%%%%%%%%%%%


\def\includegraphic{}
\def\includegraphics{}



%%% Put your definitions there:
\startlocaldefs
\endlocaldefs

\usepackage{hyperref}

%%% Begin ...
\begin{document}

%%% Start of article front matter
\begin{frontmatter}

\begin{fmbox}
\dochead{Research}

%%%%%%%%%%%%%%%%%%%%%%%%%%%%%%%%%%%%%%%%%%%%%%
%%                                          %%
%% Enter the title of your article here     %%
%%                                          %%
%%%%%%%%%%%%%%%%%%%%%%%%%%%%%%%%%%%%%%%%%%%%%%

\title{Lossy compression of DNA sequencing quality
  data}

%%%%%%%%%%%%%%%%%%%%%%%%%%%%%%%%%%%%%%%%%%%%%%
%%                                          %%
%% Enter the authors here                   %%
%%                                          %%
%% Specify information, if available,       %%
%% in the form:                             %%
%%   <key>={<id1>,<id2>}                    %%
%%   <key>=                                 %%
%% Comment or delete the keys which are     %%
%% not used. Repeat \author command as much %%
%% as required.                             %%
%%                                          %%
%%%%%%%%%%%%%%%%%%%%%%%%%%%%%%%%%%%%%%%%%%%%%%

\author[
   addressref={aff1,aff2},                   % id's of addresses, e.g. {aff1,aff2}
%   noteref={n1},                        % id's of article notes, if any
   email={chrismh@uw.edu}   % email address
]{\inits{CMH}\fnm{Christopher M.} \snm{Hill}}
\author[
   addressref={aff4,aff6},
   email={szolek@informatik.uni-tuebingen.de}
]{\inits{AS}\fnm{Andr\'{a}s} \snm{Szolek}}
\author[
   addressref={aff5,aff6},
   email={mohamed.el-hadidi@uni-tuebingen.de}
]{\inits{MEH}\fnm{Mohamed} \snm{El Hadidi}}
\author[
   addressref={aff3},
   corref={aff2},                       % id of corresponding address, if any
   email={mike@umiac.umd.edu}
]{\inits{MPC}\fnm{Michael P.} \snm{Cummings}}

%%%%%%%%%%%%%%%%%%%%%%%%%%%%%%%%%%%%%%%%%%%%%%
%%                                          %%
%% Enter the authors' addresses here        %%
%%                                          %%
%% Repeat \address commands as much as      %%
%% required.                                %%
%%                                          %%
%%%%%%%%%%%%%%%%%%%%%%%%%%%%%%%%%%%%%%%%%%%%%%

\address[id=aff1]{%                           % unique id
  \orgname{Department of Computer Science, University of Maryland}, % university, etc
  \city{College Park, Maryland},                              % city
  \postcode{20742}                                % post or zip code
  \cny{USA}                                    % country
}
\address[id=aff2]{%                           % unique id
  \orgname{Department of Genome Science, University of Washington School of Medicine},                    %
  \city{Seattle, Washington},                              % city
  \postcode{98195},                                % post or zip code
  \cny{USA}                                    % country
}
\address[id=aff3]{%                           % unique id
  \orgname{Center for Bioinformatics and Computational Biology, University of Maryland}, % university, etc
  \city{College Park, Maryland},                              % city
  \postcode{20742}                                % post or zip code
  \cny{USA}                                    % country
}
\address[id=aff4]{%
  \orgname{Department of Applied Bioinformatics, Center for
  Bioinformatics, Quantitative Biology Center, University of T\"{u}bingen},
  \street{Sand 14},
  \postcode{72076}
  \city{T\"{u}bingen},
  \cny{Germany}
}
\address[id=aff5]{%
  \orgname{Department of Algorithms in Bioinformatics, Center for
  Bioinformatics, Quantitative Biology Center, University of T\"{u}bingen},
  \street{Sand 14},
  \postcode{72076}
  \city{T\"{u}bingen},
  \cny{Germany}
}
\address[id=aff6]{%
  \orgname{Department of Computer Science, University of T\"{u}bingen},
  \street{Sand 14},
  \postcode{72076}
  \city{T\"{u}bingen},
  \cny{Germany}
}

%%%%%%%%%%%%%%%%%%%%%%%%%%%%%%%%%%%%%%%%%%%%%%
%%                                          %%
%% Enter short notes here                   %%
%%                                          %%
%% Short notes will be after addresses      %%
%% on first page.                           %%
%%                                          %%
%%%%%%%%%%%%%%%%%%%%%%%%%%%%%%%%%%%%%%%%%%%%%%

%\begin{artnotes}
%\note{Sample of title note}     % note to the article
%\note[id=n1]{Current address: \ldots} % note, connected to author
%\end{artnotes}

\end{fmbox}% comment this for two column layout

%%%%%%%%%%%%%%%%%%%%%%%%%%%%%%%%%%%%%%%%%%%%%%
%%                                          %%
%% The Abstract begins here                 %%
%%                                          %%
%% Please refer to the Instructions for     %%
%% authors on http://www.biomedcentral.com  %%
%% and include the section headings         %%
%% accordingly for your article type.       %%
%%                                          %%
%%%%%%%%%%%%%%%%%%%%%%%%%%%%%%%%%%%%%%%%%%%%%%

\begin{abstractbox}

\begin{abstract} % abstract
\parttitle{Motivation:} %if any
As the cost of sequencing continues to decrease, the rate of sequence
data production is increasing, placing greater demands on storing and
transferring these vast amount of data. Most methods of sequencing
data compression focus on compressing nucleotide information without
any loss of information. Quality data, however, have different
properties than nucleotide data, and methods compressing nucleotide
sequences efficiently do not perform as well on quality
values. Furthermore, although lossless representation might be
necessary for nucleotide sequences, it is not an essential requirement
for quality values.

Previous studies of quality value compression have mostly focus on
minimizing the loss of information with less emphasis on the effects
on bioinformatic analyses. In this paper, we evaluate several
different compression methods for quality values, and assess the
resulting impacts on common bioinformatic analyses using sequence read
data: quality control, genome assembly, and alignment of short reads
to a reference sequence.

\parttitle{Results:} %if any
Lossy compression of quality information can greatly decrease memory
requirements, and our results demonstrate that some compression
methods can result in transformed quality values that are quite
useful, and in some cases advantageous, compared to original
uncompressed values.
\end{abstract}

%%%%%%%%%%%%%%%%%%%%%%%%%%%%%%%%%%%%%%%%%%%%%%
%%                                          %%
%% The keywords begin here                  %%
%%                                          %%
%% Put each keyword in separate \kwd{}.     %%
%%                                          %%
%%%%%%%%%%%%%%%%%%%%%%%%%%%%%%%%%%%%%%%%%%%%%%

\begin{keyword}
\kwd{sample}
\kwd{article}
\kwd{author}
\end{keyword}

% MSC classifications codes, if any
%\begin{keyword}[class=AMS]
%\kwd[Primary ]{}
%\kwd{}
%\kwd[; secondary ]{}
%\end{keyword}

\end{abstractbox}
%
%\end{fmbox}% uncomment this for twcolumn layout

\end{frontmatter}

%%%%%%%%%%%%%%%%%%%%%%%%%%%%%%%%%%%%%%%%%%%%%%
%%                                          %%
%% The Main Body begins here                %%
%%                                          %%
%% Please refer to the instructions for     %%
%% authors on:                              %%
%% http://www.biomedcentral.com/info/authors%%
%% and include the section headings         %%
%% accordingly for your article type.       %%
%%                                          %%
%% See the Results and Discussion section   %%
%% for details on how to create sub-sections%%
%%                                          %%
%% use \cite{...} to cite references        %%
%%  \cite{koon} and                         %%
%%  \cite{oreg,khar,zvai,xjon,schn,pond}    %%
%%  \nocite{smith,marg,hunn,advi,koha,mouse}%%
%%                                          %%
%%%%%%%%%%%%%%%%%%%%%%%%%%%%%%%%%%%%%%%%%%%%%%

%%%%%%%%%%%%%%%%%%%%%%%%% start of article main body
% <put your article body there>

%%%%%%%%%%%%%%%%
%% Background %%
%%
\section*{Background}
Read data from high-throughput sequencing constitutes the largest
category of data in genomics research because of great redundancy,
inclusion of quality values, and read-level naming and
metadata. Because of this abundance, effective compression of read
data has the potential to substantially improve data storage and
transfer efficiency.

Quality values comprise a standard component of \textsc{fastq}
files~\cite{Cock:2010ve}, a very common format for sequence read
data. At the level of the sequence read, the probability of error for
each base-call is typically represented by a \textsc{phred} quality
value, which is defined as $Q =
-10\,log_{10}P$~\cite{Ewing:1998ly}. Depending on the sequencing
technology, these quality values can range from 0 to 93, and are
represented with the \textsc{ascii} characters 33 to 126 (with some
offset). There is a single quality value per base-call for Illumina
sequence reads.

Quality values can be used throughout bioinformatics pipelines. Among
the most fundamental uses of sequence quality values is as part of the
quality assessment and quality control (\textsc{qa/qc}) processes
prior to subsequent analysis steps. Quality control based on quality
values generally includes two operations: \textit{i}.~filtering, which
removes reads that on the whole do not meet arbitrary quality
standards, thus reducing the total number of reads; and
\textit{ii}.~trimming of low-quality base-calls from reads, which
reduces the length of reads trimmed. Quality values can be used by
genome assembly software to produce better
assemblies~\cite[e.g.,][]{Bryant:2009uq,Gnerre:2011kx}. Short-read
alignment programs, such as Bowtie2~\cite{Langmead:2012rw}, use
quality values to weight mismatches between read and reference
sequences. Software for detecting single nucleotide polymorphisms
(\textsc{snp}s) can use quality values~\cite[e.g.,][]{McKenna:2010bh},
and identified \textsc{snp}s with high-quality calls are deemed more
reliable than those with low-quality calls, particularly in
low-coverage regions.

Previous literature on sequence data compression has largely focused
on lossless compression of base calls~\cite[reviewed
  in][]{Deorowicz:2013hq, Giancarlo:2009fk,Giancarlo:2014rw,
  Nalbantoglu:2010uq,Zhu:2013qr}, although some recent work has
addressed compression of quality
values~\cite[e.g.,][]{Canovas:2014fr,Hach:2012ys,
  janin2013adaptive,Kozanitis:2011kl,Ochoa:2013rt,Tembe:2010ys,
  Wan:2012kq,DBLP:conf/recomb/YuYB14,zhou2014compression,Malysa01102015}.
Among the several challenges for compression of read data is dealing
with different error profiles resulting from differences in underlying
chemistries, signal detection and processing mechanisms, inherent
biases, and other idiosyncratic properties of distinct high-throughput
sequencing technologies.

Although we recognize the need for lossless compression for some
purposes and contexts (e.g., archiving, provenance), our perspective
is largely pragmatic with a focus on the use of quality values in
bioinformatic analyses. From this perspective, some loss of
information is deemed acceptable if the inferences from analyses are
relatively unaffected. Sequence reads generated by instruments such as
an Illumina HiSeq, the focus of this research due to its preponderance
in genomics, are characterized by having relatively few insertion and
deletion errors, but substitution (miscall) errors are much more
frequent and have context-specific patterns. These errors are
non-uniformly distributed over the read length (e.g., error rates
increase up to $\sim$16$\times$ at the 3$^{\prime}$ end, and 32.8 --
67.9\% of reads have low-quality tails at the 3$^{\prime}$
end~\cite{Minoche:2011km}). Recognizing these properties of Illumina
sequence reads motivates our exploration of three general classes of
lossy compression methods -- binning, modeling, and profiling -- and
our consideration of an exemplar of each class. Some studies have
evaluated the effects of lossy compression on identifying variants
within a data set~\cite{Canovas:2014fr,janin2013adaptive}. Here we
describe our research investigating lossy compression of sequence read
quality values, specifically those associated with Illumina
instruments, with the objective to provide some perspective on several
strategies rather than to develop robust high-quality software for
use. We assess the effects of quality value information loss resulting
from compression on \textsc{dna} sequence data analyses, including
read preprocessing (filtering and trimming), genome assembly, and read
mapping.

\section*{Methods}

\subsection*{Compression strategies}

\subsubsection*{Binning}

Quality values can be binned, and the minimum number of bins that
allows for any distinction among quality values is two; i.e., two
categories of quality: ``good'' and ``bad''. We implement 2-bin
encoding by setting a quality value threshold empirically determined
by the distribution of quality values across reads. Base-calls are
marked ``bad'' if their quality value falls below the first quartile
minus 1.5 $\times$ the interquartile range (\textsc{iqr}), which is
the difference between the first and third quartile; 1.5 $\times$
\textsc{iqr} is the value used by Tukey's box
plot~\cite{mcgill1978variations}. The main benefit of this approach is
that it is completely data-dependent, and no assumptions regarding the
distribution of the quality values need to be made. We adopt the
quality value assignments of the read preprocessing tool
Sickle~\cite{sickle}, which correspond to the 40 for ``good'' and 10
for ``bad''.

With 2-bin encoding, binary encoding is possible, allowing us to use a
single bit to represent the quality of a base instead of the standard
8 bits used to store quality values in \textsc{ascii}. An additional
possible benefit of 2-bin encoding is for increased adjacency of
identical values and repeating patterns, properties that may increase
effectiveness of subsequent compression using established
algorithms~\cite{HUFFMAN:1952nr,Ziv77auniversal,
  DBLP:journals/tit/ZivL78}, although this potential benefit is not
evaluated in this study.

The economic costs of file size for binning, in general terms, include
no fixed costs, and marginal costs that are the product of the number
of base-call quality values times the cost of the encoding.

Previous work by others \cite{Wan:2012kq} describe three similar lossy
compression strategies based on binning the base error probability
distribution: UniBinning, Truncating, and LogBinning. UniBinning
evenly splits the error probability distribution into a user-defined
number of partitions.  Truncating treats a user-defined number of
highest quality values as a single bin. LogBinning works similarly to
UniBinning, except it uses the \emph{log} of the error probability
distribution, which effectively bins the \textsc{ascii} quality values
evenly. The 2-bin encoding examined here can be viewed as something
like a combination of LogBinning and Truncating in that we place the
highest quality values (as defined above) from the distribution of log
error probability values into a single bin.

Although we focus only on two bins in this work, a greater number of
bins can be used in practice, albeit with a higher cost in file size.
For example, an additional bin, thus giving bins of good, bad, and
intermediate, could be used for quality values near the border between
bins in the 2-bin encoding. Additional bins beyond two, and their
resulting compressibility and effect on downstream analyses, are not
considered here.

\subsubsection*{Modeling}

If quality values are modeled, compression is conceivably possible by
replacing the original quality values by a representation of the
model. For example, quality values can be conceptualized as bivariate
data with the ordered nucleotides (1 to read length) representing the
abscissa, and quality values representing the ordinate. We model read
quality values as polynomial functions obtained with least-squares
fitting, as one approach to the compression of read quality values by
modeling. Although polynomial functions have significantly fewer
parameters than a read-length string of raw quality values (e.g., one
to eight coefficients in our approach), the use of double-precision
floating-point numbers to store coefficients greatly limits the
compression potential of the models. The coefficients could be
represented with fewer bits, perhaps with some loss of precision, but
we have not considered alternative representations in this study.

The economic costs of file size for model-based compression, in
general terms, include no fixed costs, and marginal costs that are the
product of the number of reads times the cost of representing the
model parameters.

\subsubsection*{Profiling}

Sets of strings representing quality values show similar trends over
their length, and it is possible to identify common patterns in the
data and use them as reference profiles to approximate individual
sequences of quality values. Here we use $k$-means clustering, a
vector quantization method that partitions a set of samples into $k$
sets that minimize within-cluster sum of
squares~\cite{macqueen1967some}. We sampled 1 $\times$ 10$^{4}$ reads
at random and computed cluster centers as read quality profiles using
a heuristic iterative refinement approach that quickly converges to a
locally optimal minimum~\cite{hartigan1979algorithm}. All reads are
evaluated and the nearest quality profile in Euclidean space being
assigned to every read as their compressed representation.  The
compressed quality file therefore consists of an index enumerating the
$k$ quality profiles, and a binary part containing the assigned
quality profile index for each read.  Although this approach is not
able to capture some read-specific differences in quality values, it
does approximate the overall patterns in quality values. An example of
128 quality profiles is shown in Figure \ref{fig:profiles_128}.

The economic costs of file size for profile-based compression, in
general terms, include fixed costs associated with representing the
profiles, which is the product of the number of profiles times the
cost of encoding them. These fixed costs are amortized over the entire
set of reads to which they apply, and thus on a read basis vary as a
reciprocal function of the number of reads. Additionally there are
marginal costs that are the product of the number of reads encoded
times the cost of the encoding. Profiling like modeling, but not
binning, has the advantageous property that marginal costs are the
reciprocal of the read length. Thus the costs of file size for
profile-based compression decrease proportionally with increases in
either or both the number of reads and read length.

Profiles can be obtained from techniques other than $k$-means
clustering. For example, we could store the profiles of polynomial
functions to increase the compressibility of polynomial regression
modeling. In other words, we could use a spline (a function that is
piecewise-defined by polynomial functions) to represent quality
values. However, we have not explored this or other approaches here.

\subsection*{Data sets}

We used several Illumina sequence read data sets taken from data from
the \textsc{gage} (Genome Assembly Gold-Standard
Evaluations)~\cite{Salzberg:2012rc}, except as noted. These data sets
are as follows.

\textit{Rhodobacter sphaeroides} 2.4.1, which are generated from a
fragment library (insert size of 180 bp; 2,050,868 paired-end reads;
read length 101 nt) and short-jump library (insert size of 3,500 bp;
2,050,868 reads; read length 101 nt). The corresponding reference
sequence was obtained from the \textsc{ncbi} RefSeq database
(NC\_007488.1, NC\_007489.1, NC\_007490.1, NC\_007493.1, NC\_007494.1,
NC\_009007.1, NC\_009008.1).

\textit{Homo sapiens} chromosome 14 data, which are generated from a
fragment library (insert size of 155 bp; 36,504,800 paired-end reads)
and short-jump library (insert sizes ranging from 2283-2803 bp;
22,669,408 reads; read length 101 nt). The corresponding reference
sequence was obtained from the \textsc{ncbi} RefSeq database
(NC\_000014.8).

\textit{Escherichia coli} str. K-12 MG1655 MiSeq data was downloaded
from \url{http://www.illumina.com/systems/miseq/scientific_data.html},
which are generated from a fragment library (insert size of 180 bp;
1,145,8940 paired-end reads; read length 151 nt). The corresponding
reference sequence was obtained from the \textsc{ncbi} RefSeq database
(NC\_000913.2).

\textit{Mus musculus} data was downloaded from
\url{http://trace.ddbj.nig.ac.jp/DRASearch/run?acc=SRR032209}
(18,828,274 reads; length 36 nt).

\subsection*{Comparison to other methods}

For comparison to other developed methods for lossy compression of
quality values we analyzed the reference data with
\textsc{QualComp}~\cite{Ochoa:2013rt},
\textsc{rqs}~\cite{DBLP:conf/recomb/YuYB14}, and
\textsc{qvz}~\cite{Malysa01102015}.

\textsc{QualComp} models quality values as a multivariate Gaussian
distribution, computing the mean and covariance for each position in
the read, and these values are stored by the decoder to later
reconstruct a \emph{representative} quality value. \textsc{QualComp}
takes as input a user-specified rate (bits/read), and these bits are
apportioned among positions within the read to minimize the average
error. The quality values can be clustered beforehand to produce more
accurate models.

The economic costs of file size for compression with
\textsc{QualComp}, in our interpretation, appear to have fixed costs
for storing the cluster representatives, and the marginal costs that
are the product of the number of reads times the cost of representing
the model parameters.

\textcolor{red}{Add brief description of \textsc{rqs} and
  \textsc{qvz}, including economic costs.}

\textsc{qvz} extends the idea of using rate-distortion theory introduced in \textsc{QualComp}, allowing additional distortion metrics besides \textsc{mse} to be minimized with respect to user-specified rate (bits/read).

\textsc{rqs} first constructs a dictionary of commonly occurring $k$-mers from either the sequencing data set they seek to compress or a larger population-sized sequencing data set.
During compression, $k$-mers within reads that have a small Hamming distance from a dictionary $k$-mer are identified.
These positions may correspond to SNPs or sequencing error.
Positions within the $k$-mer that differ from the nearest $k$-mer in the dictionary are marked.
Lastly, the quality values of the remaining positions are smoothed.



\subsection*{Subsequent (secondary) compression}

We secondarily compress all otherwise compressed data sets using the
Burrows-Wheeler algorithm~\cite{bwt} via \textsc{bz}ip2, and it is the
results of these subsequent compressions that we report. Although
other algorithms~\cite[e.g.,][]{HUFFMAN:1952nr,Ziv77auniversal,
  DBLP:journals/tit/ZivL78} might be particularly effective for
subsequent compression for some compressed data representations
examined, we not explored such possibilities here, and instead
restrict ourselves to \textsc{bz}ip2 because it is commonly used and
readily available. Similarly, it may very well be possible to obtain
increased secondary compression by reordering (e.g., sorting)
compressed representations (e.g., sets of bin designations, model, or
profiles), but these operations have also not been explored here.

\subsection*{Performance evaluation}

As a measure of compression effectiveness, which reflects the sum of
fixed and marginal costs, we use bits/base-call, defined as the size
of the compressed representation of quality values (in bits) divided
by the number of quality values represented. As a measure of
information loss we use mean squared error (\textsc{mse}) as a loss
function, and define it as $\frac{1}{n}\sum_{i=1}^{n}{(Q_i'-Q_i)^2}$,
where $n$ is the number of quality values, $Q_i'$ is the
compressed/decompressed quality value, and $Q_i$ is the original
quality value associated with position $i$.

We evaluate effects of information loss from quality value compression
on quality control steps of trimming and read filtering, which were
performed using Sickle~\cite{sickle}, and make comparison to original
uncompressed data. Sickle starts at the ends of the read and uses a
sliding window (0.1 $\times$ the read length) to find locations where
the mean quality in the window falls below a given threshold (20 by
default). The sequence is then trimmed from this position to the read
end. If the trimmed sequence length is less than a certain threshold
(20 by default), then the sequence is discarded.

We evaluate effects of information loss from quality value compression
on \emph{de novo} genome assembly performance using contiguity
statistics, log average read probability
(\textsc{lap})~\cite{Ghodsi:2013hb}, and a collection of
reference-based metrics. The contiguity statistics include N50, which
is defined as the median contig size (the length of largest contig $c$
such that the total size of the contigs larger than $c$ exceeds half
of the assembly size) and corrected N50, which is the recalculated N50
size after the contigs are broken apart at locations of errors. The
\textsc{lap} score can be viewed as a log-likelihood score, where a
larger value is better. We use a script provided by \textsc{gage}
reference-based evaluation to count single nucleotide polymorphisms
(\textsc{snp}s), relocations, translations, and inversions. The
reference-based metrics are normalized by the length of the assembly
to facilitate comparison. For the genome assembly we used software
that makes use of quality values in the assembly process:
\textsc{allpaths-lg}~\cite{Gnerre:2011kx} version r50191 with default
settings and 32 threads.

\section*{Results and discussion}

\subsection*{Compression effectiveness versus information loss}

We compare the quality-value \textsc{mse} versus bits/base-call of the
\textit{Rhodobacter sphaeroides}, \textit{Homo sapiens},
\textit{Escherichia coli}, and \textit{Mus musculus} data sets for
quality values resulting from the compression methods examined (Figure
\ref{fig:mse_vs_bpbp}). Here we present the fragment libraries for the
\textit{Rhodobacter sphaeroides}, and \textit{Homo sapiens} data sets;
the corresponding short-jump library results are available in
Supplemental Table 1. Storing the original uncompressed quality values
requires 1 byte/base-call because they are stored in \textsc{ascii}
format, and the lossless compression of each original data set using
\textsc{bz}ip2 ranges from 2.19--3.10 bits/base-call; these values
provide a reference point using a widely available general compression
program. Values from the same class of compression method tend to
cluster together across the different data sets, with the 0-degree
polynomial regression, profile encodings, and
\textsc{q}ual\textsc{c}omp yielding the lowest bits/base-call values.

\textsc{q}ual\textsc{c}omp with the rate parameter set to 100
bits/read has the lowest \textsc{mse}, but requires
$\sim$10--15$\times$ larger file size than the profile encoding
methods for only a $\sim$2--3$\times$ reduction in error. When the
rate parameter of \textsc{q}ual\textsc{c}omp is set to match the
profile encoding methods, \textsc{q}ual\textsc{c}omp performs slightly
worse in terms of \textsc{mse}. For the \textit{Rhodobacter
  sphaeroides} fragment library, \textsc{q}ual\textsc{c}omp with rate
set to 10 bits/read (0.099 bits/base-call) has a \textsc{mse} of
17.29, whereas 256-profile encoding only requires 0.079 bits/base-call
and has a \textsc{mse} of 11.85.

As the degree of the polynomial increases, the bits/base-call increase
and the \textsc{mse} decreases at an exponential rate. The 7th-degree
polynomial regression results in the largest bits/base-call, as it
requires 64 bytes per read before subsequent compression with
\textsc{bz}ip2. For the \textit{Mus musculus} data set, which has read
lengths of only 36 nt, the file size, even after subsequent
compression with \textsc{bz}ip2 exceeds that for the original
\textsc{ascii} quality values.

As an additional point of reference regarding compression
effectiveness, the recently published
Read-Quality-Sparsifier~\cite{DBLP:conf/recomb/YuYB14} achieved
best-case compression of 0.254 bits/base-call on \textit{Homo sapiens}
chromosome 21 data (read lengths ranging from 50--110 nt), and a mean
compression of 1.841 bits/base-call; both values are larger than all
of the profile results and for some results of other methods presented
here (Figure~\ref{fig:mse_vs_bpbp}).

\subsection*{Effects on sequence read preprocessing}

Preprocessing involves two steps: \emph{discarding} reads that are
deemed to be poor-quality overall, and \emph{trimming} the
poor-quality regions of the reads. After trimming, the majority of
compression methods resulted in retaining more base-calls than the
original quality values (Figure \ref{fig:preprocessing}). In general,
as a given compression method increases in complexity --- i.e., as the
number of profiles, polynomial degrees, or rate increases --- the
amount of base-calls retained more closely approximates the number of
base-calls retained using the original quality values. The compression
methods on the \textit{Mus musculus} data set has the greatest
proportion of retained base-calls compared to the original quality
values. The \textit{Escherichia coli} MiSeq data set has the smallest
range.

The 2-bin approach is the only compression method that results in a
larger number of trimmed base-calls compared to the original
uncompressed reads across all data sets. Sickle uses a sliding window
approach to smooth the read quality values before it trims. In the
2-bin approach, there is an uneven distribution of values per bin. In
other words, \emph{bad} quality values may range from 0--33, whereas
\emph{good} values may only range from 34--40. Thus, mid-range quality
values that are above the threshold (20 by default) are set below the
quality threshold when compressed, resulting in an increased number of
trimmed bases.

The 0-degree polynomial regression results in the highest proportion
of base-calls kept. If the mean quality value of the read is above the
filtering threshold, then no base-calls are trimmed. Only reads that
are comprised of mostly low quality values will be discarded.

It is important to highlight that even though a compression method may
result in the same number of trimmed base-calls as the uncompressed
quality values, it does not mean the \emph{same} base-calls were
retained.  For example, pre-processing of the 1st-degree and
5th-degree polynomial regression model compressed reads of the
\textit{Rhodobacter sphaeroides} fragment library retain approximately
the same number of base-calls. However, if we examine the specific
reads discarded, the 5th-degree model discards approximately
two-thirds less reads than the 1st-degree model (4,640 and 12,066
reads, respectively; Supplemental Table 2), which means there are
differences in the base-calls trimmed.

\subsection*{Effects on genome assembly}

No compression method resulted in the uniformly best assembly of the
\textit{Rhodobacter sphaeroides} data set in all metrics, although
several methods resulted in better assemblies than the original
uncompressed data (Table \ref{fig:assembly_ranks}). Among the
compression methods, the profile encoding performed best, polynomial
regression modeling performed worst, and other methods were
intermediate (Fig.~\ref{fig:assembly_ranks}).

The lossy compression methods largely preserve the contiguity found in
the assembly produced using the reads with the original quality
values. All compression methods other than 0-degree polynomial
regression produce an N50 ranging from 3.17--3.22 Mbp (see
vSupplemental Table 3). Despite the similar contiguity statistics, the
different compression methods vary markedly in the number of
\textsc{snp}s. The 2-bin and profile methods exhibited the fewest
\textsc{snp}s compared to the reference genome, thus outperforming the
assembly using the original quality values in this characteristic. A
more in-depth evaluation is needed to determine whether these
compression methods are missing actual \textsc{snp}s.

It is important to highlight that using the original uncompressed
quality values does not produce the best assembly in terms of any of
the metrics. The assembly based on the original uncompressed quality
values score worse than the top overall assembly (256-profile
encoding) for number of assembled bases, missing reference bases, N50,
\textsc{snp}s, indels $>$5bp, and relocations. The assembly using the
original uncompressed quality values has an error rate of $\sim$8.75
errors/100 kb of assembled sequence, and the 256-profile encoding has
an error rate of $\sim$8.02 errors/100 kbp (Supplemental Table 3).

In general, the greater the polynomial degree, the better the overall
assembly; however, the 5th-degree polynomial regression performs
slightly worse than the 3rd-degree polynomial. The respective ranks in
terms of N50 and relocations are fairly distant, which lowers the
overall ranking of the 5th-degree polynomial slightly below that for
the 3rd-degree polynomial model. The 1st- and 0-degree polynomial
regression methods perform poorly in all metrics except assembled
bases. One explanation for this observation is that the high-error
portions of reads are being marked as high quality, so
\textsc{allpaths-lg} is unable to trim or error-correct the
reads. Assembled sequences that overlap may be unable to align across
the errors at the end of the reads, artificially inflating the
assembly size.

Among the different \textsc{q}ual\textsc{c}omp rate parameters, the 10
bits/read rate ranked highest overall, outperforming the other rate
parameters in terms of corrected N50, fewest missing reference bases,
\textsc{snp}s, and indels $>$5bp. With the exception of the 6
bits/read rate, the assemblies decrease in rank with the increase in
the rate parameter for corrected N50, and fewest missing reference
bases. This trend runs counter to the decrease in \textsc{mse} of the
different rates.

\subsection*{Effects on read mapping}

Some short read alignment tools, such as Bowtie2 (version 2.2.3),
which was used here, utilize quality value information whenover
evaluating potential alignments. The reads from with original
uncompressed and compressed/decompressed quality values were mapped
with Bowtie2 to the \textit{Rhodobacter sphaeroides} reference
genome. The total, shared, and unique proportion of mapped reads are
calculated with respect to the results for the original uncompressed
quality values as shown in Table \ref{tab:aligner}. Additionally, to
assess the effect of quality values on mapping in general, Bowtie2 was
adjusted so that the maximum and minimum mismatch penalty were
equivalent to maximum and minimum quality scores (with parameters:
--mp 6,6 and --mp 2,2 respectively).

We evaluate the compression methods to reflect that comparisons among
mapping identical reads with different quality values may vary in two
mutually exclusive ways: reads map to the reference, and reads do not
map to a reference. Thus comparing mapping using original quality
values and reads using quality values modified via lossy compression,
we have a $2 \times 2$ contingency table consisting of reads that map
for both original and compressed, reads that map for original and do
not map for compressed, reads that do not map for original and map for
compressed, and reads that do not map for both original and
compressed.

The best compression method in terms of similarity with the
uncompressed reads is \textsc{q}ual\textsc{c}omp with rate 100
bits/read, followed by \textsc{q}ual\textsc{c}omp with rate 30
bits/read, 256-profile encoding, 128-profile encoding, 2-bin encoding,
64-profile encoding, \textsc{q}ual\textsc{c}omp with rate 10
bits/read, 7th-degree polynomial regression,
\textsc{q}ual\textsc{c}omp with rate 6 bits/read, and finally,
5th-degree through 0-degree polynomial regression.

Ranking the compression methods by overall proportion of reads aligned
produces an identical ordering as above. Aside from 0-degree
polynomial regression (0.831), all other compression methods have a
read alignment rate between 0.87 and 0.861. The proportion of reads
aligned for the uncompressed reads is 0.87.

Most of the compression methods did not vary greatly in terms of the
number of reads that were mapped \emph{only} using quality values
resulting from compression; however, there is a sizable difference in
the amount of reads that are originally mapped, but unmapped by the
compressed methods. \textsc{q}ual\textsc{c}omp with rate 100 bits/read
results in the fewest missing original read alignments
(159). Increasing the regression model polynomial degree results in a
decreasing amount of reads that are originally mapped, but unmapped by
the regression model (40,931 and 1,134 reads for 0-degree and
7th-degree, respectively). There is no such trend for reads that are
mapped only by the regression model.

Note that the 2-bin method aligns a greater portion of reads than the
various regression models. If a poor-quality base-call is flanked by
high-quality base-calls, then the low-degree polynomial regression
models tend to smooth out the quality valus, erroneously marking the
low-quality base-calls as higher quality. During alignment, Bowtie2
penalizes mismatches at high-quality base-calls more so than
mismatches at lower-quality base-calls. Thus, the polynomial
regression models incur a high penalty for these base-calls, resulting
in fewer alignments, despite having a better \textsc{mse} than 2-bin
in most cases.

Setting all base-calls as minimum quality results in the highest
proportion of mapped reads (0.881). Conversely, setting all base-calls
as maximum quality results in the lowest proportion of mapped reads
(0.728).

\section*{Conclusions}

We have examined several simple and general approaches for lossy
compression of read quality values, and their effect on bioinformatic
analyses. Our results demonstrate that some compression methods can
result in quality values that are quite useful, and in some cases
advantageous, compared to original uncompressed values.

Downstream applications dictate the relative performance lossy
compression methods.  Some bioinformatics tools proved to be robust to
moderate information loss in base-call quality. The use of quality
values modified via several compression/decompression processes result
in approximately the same total number base-calls passing through the
combined filtering and trimming processes as when using original
quality values, though the absolute and relative number of reads
filtered and base-calls trimmed may differ. Genome assembly appears
robust to standard sequencing errors, and compression transformed
quality values, particularly those associated with profiling-based
compression, result in assemblies better than those based on original
uncompressed quality values by most measures.

Among the potential benefits of compressing quality values is the
ability to perform quality control and possibly other operations
directly on the compressed representations of the data. For example,
with profile-based compression, each of the $k$ profiles can be
evaluated for (pre-)processing operations such as filtering and
trimming, and the operations transitively applied to the entire set of
reads, thus saving substantial computation.

Quality values for base-calls in \textsc{dna} sequencing are not
sacrosanct; they are representations of estimates for measurement
(detection) error determined by various mechanisms including signal
processing characteristics and algorithms. The true worth of quality
values lies in the utility they provide in bioinformatic analyses. In
some cases, as demonstrated here, this value is enhanced through
transformation associated with compression. Thus appropriate
compression of quality values can both reduce associated file size
cost and improve analysis results.


%%%%%%%%%%%%%%%%%%%%%%%%%%%%%%%%%%%%%%%%%%%%%%
%%                                          %%
%% Backmatter begins here                   %%
%%                                          %%
%%%%%%%%%%%%%%%%%%%%%%%%%%%%%%%%%%%%%%%%%%%%%%

\begin{backmatter}

\section*{Competing interests}
  The authors declare that they have no competing interests.

\section*{Author's contributions}
    Text for this section \ldots

\section*{Acknowledgements}
This project was initiated at the 2014 Bioinformatics Exchange for
Students and Teachers (\textsc{best}) Summer School, which was funded
by the offices of the Dean of The Graduate School, University of
Maryland, and the Rektor of the University of T\"{u}bingen. Additional
funding included an International Graduate Research Fellowship from
The Graduate School, University of Maryland, to CMH, a Global
Partnerships-Faculty Travel Grant from the Office of International
Affairs to MPC, and funding from the University of T\"{u}bingen.

%%%%%%%%%%%%%%%%%%%%%%%%%%%%%%%%%%%%%%%%%%%%%%%%%%%%%%%%%%%%%
%%                  The Bibliography                       %%
%%                                                         %%
%%  Bmc_mathpys.bst  will be used to                       %%
%%  create a .BBL file for submission.                     %%
%%  After submission of the .TEX file,                     %%
%%  you will be prompted to submit your .BBL file.         %%
%%                                                         %%
%%                                                         %%
%%  Note that the displayed Bibliography will not          %%
%%  necessarily be rendered by Latex exactly as specified  %%
%%  in the online Instructions for Authors.                %%
%%                                                         %%
%%%%%%%%%%%%%%%%%%%%%%%%%%%%%%%%%%%%%%%%%%%%%%%%%%%%%%%%%%%%%

% if your bibliography is in bibtex format, use those commands:
\bibliographystyle{bmc-mathphys} % Style BST file (bmc-mathphys, vancouver, spbasic).
%\bibliographystyle{vancouver} % Style BST file (bmc-mathphys, vancouver, spbasic).
\bibliography{compression}      % Bibliography file (usually '*.bib' )
% for author-year bibliography (bmc-mathphys or spbasic)
% a) write to bib file (bmc-mathphys only)
% @settings{label, options="nameyear"}
% b) uncomment next line
%\nocite{label}

% or include bibliography directly:
% \begin{thebibliography}
% \bibitem{b1}
% \end{thebibliography}

%%%%%%%%%%%%%%%%%%%%%%%%%%%%%%%%%%%
%%                               %%
%% Figures                       %%
%%                               %%
%% NB: this is for captions and  %%
%% Titles. All graphics must be  %%
%% submitted separately and NOT  %%
%% included in the Tex document  %%
%%                               %%
%%%%%%%%%%%%%%%%%%%%%%%%%%%%%%%%%%%

%%
%% Do not use \listoffigures as most will included as separate files

\section*{Figures}
\begin{figure}[h!]
% profiles128.eps
\caption{\csentence{Quality profiles obtained by $k$-means
    clustering.} Example for the fragment library from
  \textit{Rhodobacter sphaeroides} 2.4.1 data set using $k$ =
  128. Each row corresponding to a quality profile. Dark to light
  colors represent low to high quality values. It is readily visible
  that the two most distinctive features of quality profiles is their
  drop-off position and average overall quality. Occasional low values
  occur early in some profiles, likely reflecting intermittent
  problems in the sequencing process affecting many reads at a time.}
  \label{fig:profiles_128}
\end{figure}

\begin{figure*}[h!]
% compression.results.eps
\caption{\csentence{The relationship of bits/base-call and mean
    squared error for quality value compression methods applied to
    four data sets.} The data sets are \textit{Rhodobacter
    sphaeroides} 2.4.1; \textit{Homo sapiens} chromosome 14;
  \textit{Escherichia coli} str. K-12 MG1655; and \textit{Mus
    musculus}. Point labels correspond to different compression
  methods: 2B, 2-bin encoding; P$n$, profiling with $n$ profiles;
  R$n$, modeling with polynomial regression of degree $n$; Q$n$,
  \textsc{q}ual\textsc{c}omp with rate parameter of $n$. Arrows denote
  the corresponding lossless compression using \textsc{bz}ip2, with
  the black arrow corresponding to original data.}
\label{fig:mse_vs_bpbp}
\end{figure*}

\begin{figure}[h!]
% preprocessing.results.eps
\caption{\csentence{Read preprocessing results using compressed
    quality values compared original uncompressed quality values.}
  The reads are for the \textit{Rhodobacter sphaeroides} 2.4.1, and
  \textit{Homo sapiens} chromosome 14 fragment libraries, and
  \textit{Escherichia coli} str. K-12 MG1655, and \textit{Mus
    musculus} data sets. Reads were trimmed using Sickle. The total
  number of bases filtered by each compression method is compared with
  the number of bases filtered using the uncompressed quality values.}
  \label{fig:preprocessing}
\end{figure}

\begin{figure}[h!]
% rhodo.assembly.results.eps
\caption{\csentence{Rankings of compression methods based on assembly
    attributes.} The read data are for \textit{Rhodobacter
    sphaeroides} 2.4.1, and the results sorted by overall
  rank. Assemblies were constructed using \textsc{allpaths-lg}.
  Rankings above the median value are in cyan, those below the median
  value in magenta.}
  \label{fig:assembly_ranks}
\end{figure}


%%%%%%%%%%%%%%%%%%%%%%%%%%%%%%%%%%%
%%                               %%
%% Tables                        %%
%%                               %%
%%%%%%%%%%%%%%%%%%%%%%%%%%%%%%%%%%%

%% Use of \listoftables is discouraged.
%%
\section*{Tables}

\begin{table*}[!tbhp]
\caption{Comparison of mapping for original reads and reads with
  compressed/decompressed quality values. Reads and reference genome
  are for \textit{Rhodobacter sphaeroides}, and mapping was performed
  using Bowtie2.}
\begin{tiny}
\begin{tabular}{lr|cc|cc|cc|cc|cc}
 & & \multicolumn{2}{c|}{max-qual} & \multicolumn{2}{c|}{min-qual} & \multicolumn{2}{c|}{2-bin} & \multicolumn{2}{c|}{regression (0)} & \multicolumn{2}{c}{regression (1)} \\
& & mapped & unmapped & mapped & unmapped & mapped & unmapped & mapped & unmapped & mapped & unmapped \\
\cline{2-12}
& mapped & 746716 & 145897 & 892613 &   0 & 891864 & 749 & 851682 & 40931 & 883390 & 9223 \\
{\em original} & unmapped &   0 & 132821 & 10821 & 122000 & 186 & 132635 &  67 & 132754 &  55 & 132766 \\
\cline{2-12}
& proportion & 0.728 & 0.272 & 0.881 & 0.119 & 0.870 & 0.130 & 0.831 & 0.169 & 0.862 & 0.138 \\
\end{tabular}

\bigskip

\begin{tabular}{lr|cc|cc|cc|cc|cc}
 & & \multicolumn{2}{c|}{regression (3)} & \multicolumn{2}{c|}{regression (5)} & \multicolumn{2}{c|}{regression (7)} & \multicolumn{2}{c|}{profile (64)} & \multicolumn{2}{c}{profile (128)} \\
& & mapped & unmapped & mapped & unmapped & mapped & unmapped & mapped & unmapped & mapped & unmapped \\
\cline{2-12}
& mapped & 889537 & 3076 & 891019 & 1594 & 891479 & 1134 & 891753 & 860 & 891952 & 661 \\
{\em original} & unmapped & 117 & 132704 & 155 & 132666 & 154 & 132667 & 144 & 132677 & 143 & 132678 \\
\cline{2-12}
& proportion & 0.868 & 0.132 & 0.869 & 0.131 & 0.870 & 0.130 & 0.870 & 0.130 & 0.870 & 0.130 \\
\end{tabular}

\bigskip

\begin{tabular}{lr|cc|cc|cc|cc|cc}
&  & \multicolumn{2}{c|}{profile (256)} & \multicolumn{2}{c|}{\textsc{q}ual\textsc{c}omp (6)} & \multicolumn{2}{c|}{\textsc{q}ual\textsc{c}omp (10)} & \multicolumn{2}{c|}{\textsc{q}ual\textsc{c}omp (30)} & \multicolumn{2}{c}{\textsc{q}ual\textsc{c}omp (100)} \\
& &  mapped & unmapped & mapped & unmapped & mapped & unmapped & mapped & unmapped & mapped & unmapped \\
\cline{2-12}
& mapped & 892051 & 562 & 891375 & 1238 & 891777 & 836 & 892233 & 380 & 892454 & 159 \\
{\em original}  & unmapped & 119 & 132702 & 304 & 132517 & 265 & 132556 & 220 & 132601 & 172 & 132649 \\
\cline{2-12}
& proportion & 0.870 & 0.130 & 0.870 & 0.130 & 0.870 & 0.130 & 0.870 & 0.130 & 0.870 & 0.130 \\
\end{tabular}
\end{tiny}

\label{tab:aligner}
\end{table*}


%%%%%%%%%%%%%%%%%%%%%%%%%%%%%%%%%%%
%%                               %%
%% Additional Files              %%
%%                               %%
%%%%%%%%%%%%%%%%%%%%%%%%%%%%%%%%%%%

\section*{Additional Files}
  \subsection*{Additional file 1 --- Sample additional file title}
    Additional file descriptions text (including details of how to
    view the file, if it is in a non-standard format or the file extension).  This might
    refer to a multi-page table or a figure.

  \subsection*{Additional file 2 --- Sample additional file title}
    Additional file descriptions text.


\end{backmatter}
\end{document}
